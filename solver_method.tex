\documentclass[11pt, letterpaper]{article}
\usepackage[margin=1in]{geometry}
\usepackage{amsmath}

% Remove Paragraph Indent
\setlength{\parindent}{0pt}
\setlength{\parskip}{1em}


\begin{document}
{\Large \bf Numerical Solver Method} \\
{\small \it Jeffrey Rutledge}

The function titled \texttt{CrankNicolson} is the solver method intended for use, and explained here.
The other functions use different schemes.

\section*{Problem}

The PDE this solver numerically approximates is,
\[
    \frac{\partial c_g}{\partial t} = D_e \frac{\partial^2 c_g}{\partial z^2} - V_e \frac{\partial c_g}{\partial z}
\]
where $c_g$ is the concentration of the tracer in the gas phase, $z$ is the depth under the surface, $D_e$ is the effective diffusion constant, $V_e$ is the effective velocity, and $t$ is time.
The boundary at the surface, $c_g(t,\ z = 0)$, is the measured atmospheric concentration of the tracer gas.
The initial boundary, $c_g(t = 0,\ z)$, and the boundary at the max depth of 200 meters, $c_g(t,\ z = 200)$, are both assumed to be zero.

\section*{Solution}
The PDE is approximated using a scheme similar to the Crank Nicolson method, an implicit finite difference scheme.
Let $u^m_j$ be the approximation of $c_g(t, z)$ at $t = m \Delta t$ and $z = j \Delta z$.
The approximations used are forward difference in time,
\[
    \frac{\partial c_g}{\partial t} \approx \frac{u^{m + 1}_j - u^m_j}{\Delta t},
\]
and central differences in position averaged over the $(m + 1)$th time step and the $m$th time step,
\[
    \frac{\partial^2 c_g}{\partial z^2} \approx \frac{1}{2} \left(
        \frac{u^{m + 1}_{j + 1} - 2u^{m + 1}_j + u^{m + 1}_{j - 1}}{\Delta x^2} +
    \frac{u^{m}_{j + 1} - 2u^{m}_j + u^{m}_{j - 1}}{\Delta x^2} \right)
\]
and
\[
    \frac{\partial c_g}{\partial z} \approx \frac{1}{2} \left(
        \frac{u^{m + 1}_{j + 1} - u^{m + 1}_{j - 1}}{2\Delta x} +
    \frac{u^{m}_{j + 1} - u^{m}_{j - 1}}{2\Delta x} \right).
\]

When substituted into the original equation these approximations yield the equation,
\[
    \frac{u^{m + 1}_j - u^m_j}{\Delta t} = \frac{1}{2} \left(
        D_e\frac{u^{m + 1}_{j + 1} - 2u^{m + 1}_j + u^{m + 1}_{j - 1} + u^{m}_{j + 1} - 2u^{m}_j + u^{m}_{j - 1}}{\Delta x^2} -
    V_e\frac{u^{m + 1}_{j + 1} - u^{m + 1}_{j - 1} + u^{m}_{j + 1} - u^{m}_{j - 1}}{2\Delta x} \right).
\]
This can be and rearranged into the equation,
\begin{equation}
    \label{eq:iterative_solution}
    \begin{split}
        &(-2D_e - V_e\Delta x)u^{m + 1}_{j -1} + 4\left(\frac{\Delta x^2}{\Delta t} + D_e\right)u^{m + 1}_j + (-2D_e + V_e \Delta x) u^{m + 1}_{j + 1} = \\
        &\qquad\qquad (2D_e + V_e\Delta x)u^{m}_{j -1} + 4\left(\frac{\Delta x^2}{\Delta t} - D_e\right)u^{m}_j + (2D_e - V_e \Delta x) u^{m}_{j + 1}.
    \end{split}
\end{equation}
This gives a system of equations that may be solved using a tridiagonal matrix.

\section*{Accuracy}

The depth approximations are both central differences, which are second order, so their error is $O(\Delta x^2)$.
The time approximation is a forward difference which is only first order, but because the depth approximations are averaged over the $(m + 1)$th and the $m$th time steps the error in time is $O(\Delta t^2)$.
Thus the approximation is second order in the time and depth steps.


\section*{Stability}

To use von Neumann stability we will make the substitution,
\[
    u^m_j = Q^me^{ij\Delta x k},
\]
where $Q^m$ is the magnitude at the $m$th time step, $k$ is frequency, and $i$ is $\sqrt{-1}$.

Substituting this into the equation (\ref{eq:iterative_solution}),
\[
    \begin{split}
        &(-2D_e - V_e\Delta x)Q^{m + 1}e^{i(j - 1)k} + 4\left(\frac{\Delta x^2}{\Delta t} + D_e\right)Q^{m + 1}e^{ijk} + (-2D_e + V_e \Delta x) Q^{m + 1}e^{i(j + 1)k} = \\
        &\qquad\qquad (2D_e + V_e\Delta x)Q^me^{i(j - 1)k} + 4\left(\frac{\Delta x^2}{\Delta t} - D_e\right)Q^me^{ijk} + (2D_e - V_e \Delta x) Q^me^{i(j + 1)k}.
    \end{split}
\]
This can be simplified into,
\[
    Q^{m + 1} = Q^m a,
\]
where $a$ is the amplification factor,
\[
    a = \frac{(2D_e + V_e\Delta x)e^{-i\Delta xk} + (2D_e - V_e\Delta x)e^{i\Delta xk} + 4\frac{\Delta x^2}{\Delta t} - 4D_e}
    {(-2D_e - V_e\Delta x)e^{-i\Delta xk} + (-2D_e + V_e\Delta x)e^{i\Delta xk} + 4\frac{\Delta x^2}{\Delta t} + 4D_e}.
\]
Now we can reduce this further using Euler's formula to,
\[
    a = \frac{4D_e\cos\Delta xk - 2V_e\Delta xi\sin \Delta x k + 4\frac{\Delta x^2}{\Delta t} - 4D_e}
    {-4D_e\cos\Delta xk + 2V_e\Delta xi\sin \Delta x k + 4\frac{\Delta x^2}{\Delta t} + 4D_e}
\]
and finally to,
\[
    a = \frac{1 - \frac{D_e\Delta t}{\Delta x^2} (1 - \cos\Delta x k)}
    {1 + \frac{D_e\Delta t}{\Delta x^2} (1 - \cos\Delta x k)} - i.
\]
From here it can be seen that the magnitude of $a$ will always be less than or equal to 1.
Thus this method is unconditionally stable.

\end{document}
